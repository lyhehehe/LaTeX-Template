\documentclass{article}
\usepackage{geometry}
\usepackage{amsmath}
\usepackage{amsfonts}
\usepackage{ctex}
\usepackage{amssymb}
\usepackage{graphicx}
\usepackage{fontawesome}
\usepackage{xcolor}
\usepackage{listings}
\usepackage[colorlinks]{hyperref}  %用于插入超链接
\usepackage[most]{tcolorbox}
\usepackage{ifthen}
\usetikzlibrary{shadows}
\tcbset{
	myexample/.style={
		enhanced,
		width=\linewidth,
		colback=white, % 背景颜色 red!5!white
		colframe=gray!20, % 外框的颜色
		fonttitle=\bfseries,
		breakable,
		arc=2pt,
		drop shadow={gray!15,opacity=1},
		titlerule=0pt,
		title style={fill=white},
		coltitle=gray,
		drop shadow,
		highlight math style={reset,colback=white,colframe=black}
	}
}
\newtcolorbox{sBox}{myexample}
\definecolor{macosbox@red}{RGB}{236,96,92}
\definecolor{macosbox@yellow}{RGB}{247,188,44}
\definecolor{macosbox@green}{RGB}{88,204,65}

\definecolor{macosbox@top}{RGB}{237,237,237}
\definecolor{macosbox@bot}{RGB}{189,189,189}
\definecolor{macosbox@bord}{RGB}{182,176,176}
\definecolor{macosbox@bg}{RGB}{240,240,240}
\definecolor{macosbox@textcol}{RGB}{50,50,50}

\definecolor{macosbox@topdark}{RGB}{96,96,98}
\definecolor{macosbox@botdark}{RGB}{44,45,47}
\definecolor{macosbox@borddark}{RGB}{13,13,15}
\definecolor{macosbox@bgdark}{RGB}{55,56,58}
\definecolor{macosbox@textcoldark}{RGB}{223,224,226}
%%%%%%%%%%%%%%%%%%%%%%%%%%%%%%%%%%%%%%%%%%%%%%%%%%%%%%%%%%%%%%
\newenvironment{macosbox}[2][]{
	\begin{tcolorbox}[enhanced,
		coltitle=black,
		colback=macosbox@bg,
		boxrule=0mm,
		frame style={draw=macosbox@bord,fill=macosbox@bord},
		title style={top color=macosbox@top,bottom color=macosbox@bot},
		drop fuzzy shadow=black,
		#1, %hbox
		title=\hspace*{-3mm}%
		\macosbox@dot{macosbox@red} %
		\macosbox@dot{macosbox@yellow} %
		\macosbox@dot{macosbox@green}%
		\hspace*{\fill}\hspace*{-10mm}#2\hspace*{\fill}]
		\ifthenelse{\isundefined{\usemintedstyle}}{}{\usemintedstyle{\macosbox@mintedstyle}}
		\color{macosbox@textcol}
	}{
	\end{tcolorbox}
}
%%%%%%%%%%%%%%%%%%%%%%%%%%%%%%%%%%%%%%%%%%%%%%%%%%%%%%%%%%%%%%
\newtcolorbox{macbox}[2][]{%
	enhanced,
	coltitle=black,
	colback=macosbox@bg,
	boxrule=0mm,
	frame style={draw=macosbox@bord,fill=macosbox@bord},
	title style={top color=macosbox@top,bottom color=macosbox@bot},
	drop fuzzy shadow=black,
	title={{\textcolor[RGB]{236, 96, 92}{\faCircle}
		\textcolor[RGB]{247, 188, 44}{\faCircle} 
		\textcolor[RGB]{88, 204, 65}{\faCircle}
		\hspace*{\fill}\hspace*{-10mm}\texttt{#2}\hspace*{\fill}}},#1
}
\newtcolorbox{macboxd}[2][]{%
enhanced,
coltitle=white,
colback=macosbox@bgdark,
boxrule=0mm,
frame style={draw=macosbox@borddark,fill=macosbox@borddark},
title style={top color=macosbox@topdark,bottom color=macosbox@botdark},
drop fuzzy shadow=black,
	title={{\textcolor[RGB]{236, 96, 92}{\faCircle}
			\textcolor[RGB]{247, 188, 44}{\faCircle} 
			\textcolor[RGB]{88, 204, 65}{\faCircle}\hspace*{\fill}\hspace*{-10mm}\texttt{#2}\hspace*{\fill}}},
#1
}


%%%%%%%%%%%%%%%%%%%%%%%%%%%%%%%%%%%%%%%%%%%%%%%%%%%%%
\begin{document}
\begin{sBox}
\begin{center}
	\noindent\footnotesize\begin{tabular}{@{}l@{ }l|l@{ }l@{}}
		\textcolor[RGB]{18,183,245}{\faQq}:&910014191:  这是QQ!一起搞事情!\faSendO   & \textcolor[RGB]{9,187,7}{\faWeixin}:&\verb|雨霓同学|  公众号,知识与资源分享\faSendO \\
		\textcolor[RGB]{236, 29, 152}{\faWeibo}\textbf{:} & \href{https://weibo.com/u/5713129191}{微博} \faSendO 多多关注!
		 &
		 {\textcolor[RGB]{39,165,188}{\faGithubAlt}}:& \href{https://github.com/Azure1210/}{Github}  什么都没写,先放这里了 \\
		\textcolor[RGB]{0,194,255}{\faInternetExplorer}:& \href{https://www.cnblogs.com/1210x1184/}{ 博客园,技术分享} 
		 &\textcolor[RGB]{255, 148, 209}{\faUsers} &1045381079\textbf{:}  QQ群,欢迎水群!\faSendO \\[2pt]\\
		\multicolumn{4}{c}{\textcolor[RGB]{252,74,35}{\faTv}: \url{https://space.bilibili.com/44523572} \faSendO 感谢各位大佬一键三联\faSendO}\\
\end{tabular}
\end{center}	
\end{sBox}	
\begin{macbox}{A macOS window-like tcolorbox}
\begin{tabular}{@{}l@{ }l|l@{ }l@{}}
	\textcolor[RGB]{18,183,245}{\faQq}:&910014191:  这是QQ!一起搞事情!\faSendO   & \textcolor[RGB]{9,187,7}{\faWeixin}:&\verb|雨霓同学|  公众号,知识与资源分享\faSendO \\
	\textcolor[RGB]{0,194,255}{\faInternetExplorer}:& \href{https://www.cnblogs.com/1210x1184/}{ 博客园,技术分享}  & {\textcolor[RGB]{39,165,188}{\faGithubAlt}}:& \href{https://github.com/Azure1210/}{Github}  什么都没写,先放这里了 \\
	\textcolor[RGB]{236, 29, 152}{\faWeibo}\textbf{:} & \href{https://weibo.com/u/5713129191}{微博} \faSendO 多多关注! &\textcolor[RGB]{255, 148, 209}{\faUsers} &1045381079\textbf{:}  QQ群,欢迎水群!\faSendO \\[2pt]\\
	\multicolumn{4}{c}{\textcolor[RGB]{252,74,35}{\faTv}: \url{https://space.bilibili.com/44523572} \faSendO 感谢各位大佬一键三联\faSendO}\\
\end{tabular}
\end{macbox}
\begin{macboxd}[coltext=white]{A macOS window-like tcolorbox}
\begin{tabular}{@{}l@{ }l|l@{ }l@{}}
		\textcolor[RGB]{18,183,245}{\faQq}:&910014191:  这是QQ!一起搞事情!\faSendO   & \textcolor[RGB]{9,187,7}{\faWeixin}:&\verb|雨霓同学|  公众号,知识与资源分享\faSendO \\
		\textcolor[RGB]{0,194,255}{\faInternetExplorer}:& \href{https://www.cnblogs.com/1210x1184/}{ 博客园,技术分享}  & {\textcolor[RGB]{39,165,188}{\faGithubAlt}}:& \href{https://github.com/Azure1210/}{Github}  什么都没写,先放这里了 \\
		\textcolor[RGB]{236, 29, 152}{\faWeibo}\textbf{:} & \href{https://weibo.com/u/5713129191}{微博} \faSendO 多多关注! &\textcolor[RGB]{255, 148, 209}{\faUsers} &1045381079\textbf{:}  QQ群,欢迎水群!\faSendO \\[2pt]\\
		\multicolumn{4}{c}{\textcolor[RGB]{252,74,35}{\faTv}: \url{https://space.bilibili.com/44523572} \faSendO 感谢各位大佬一键三联\faSendO}\\
	\end{tabular}
\end{macboxd}	
	
\textbf{大佬们写的我编译不出来,会写宏包就是牛逼,所以我抄了大佬的代码,不过我抄对了!,内容什么的都是瞎写的信息!欢迎下载!}

\textbf{有问题直接下方留言!应该会看的,格式什么的没有进行分离,大佬别嘲笑!好多东西都不会!}.
	
	
\end{document}